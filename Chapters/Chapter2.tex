\chapter{Introducción Específica} % Main chapter title

\label{Chapter2}

%----------------------------------------------------------------------------------------
%	SECTION 1
%----------------------------------------------------------------------------------------

\section{Título de la sección con este uso de las mayúsculas}
\label{sec:ejemplo}
La idea de esta sección es presentar el tema de modo que cualquier persona que no conoce el tema pueda entender de qué se trata y por qué es importante realizar este trabajo y cuál es su impacto.

Si en el texto se hace alusión a diferentes partes del trabajo referirse a ellas como Capítulo, Sección o subsección según corresponda. Por ejemplo: ``En el Capítulo \ref{Chapter1} se explica tal cosa'', o ``En la Sección \ref{sec:ejemplo} se presenta lo que sea'', o ``En la la subsección \ref{subsec:ejemplo} se discute otra cosa''.

Entre párrafos sucesivos dejar un espacio, como el que se observa entre este párrafo y el anterior. Pero las oraciones de un mismo párrafo van en forma consecutiva, como se observa acá. Luego, cuando se quiere poner una lista tabulada se hace así:

\begin{itemize}
	\item Este es el primer elemento de la lista.
	\item Este es el segundo elemento de la lista.
\end{itemize}

Notar el uso de las mayúsculas y el punto al final de cada elemento.
Si se desea poner una lista numerada el formato es este:
\begin{enumerate}
	\item Este es el primer elemento de la lista.
	\item Este es el segundo elemento de la lista.
\end{enumerate}

Notar el uso de las mayúsculas y el punto al final de cada elemento.

\subsection{Este es el título de una subsección}
\label{subsec:ejemplo}

Se recomienda no utilizar \textbf{texto en negritas} en ningún párrafo, ni tampoco \underline{texto subrayado}. En cambio sí se sugiere utilizar \textit{texto en cursiva} donde se considere apropiado.

Se sugiere que la escritura sea impersonal. Por ejemplo, no utilizar ``el diseño del firmware lo hice de acuerdo con tal principio'', sino ``el firmware fue diseñado utilizando tal principio''. En lo posible hablar en tiempo pasado, ya que la memoria describe un trabajo que ya fue realizado.

Se recomienda no utilizar una sección de glosario sino colocar la descripción de las abreviaturas como parte del mismo cuerpo del texto. Por ejemplo, RTOS (\textit{Real Time Operating System}, Sistema Operativo de Tiempo Real) o en caso de considerarlo apropiado mediante notas a pie de página.

Si se desea indicar alguna página web utilizar el siguiente formato de referencias bibliográficas, dónde las referencias se detallan en la sección de bibliografía de la memoria,utilizado el formato establecido por IEEE en [1]. Por ejemplo, ``el presente trabajo se basa en la plataforma EDU-CIAA-NXP, la cual se describe en detalle en [2]''. 
